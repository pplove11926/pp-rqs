\section{Problem Formulation}
\label{sec:formulation}

This section formulates the problem studied and gives an overview of our solution.


\subsection{Data Model}
We consider {\em directed
  node-labeled graphs}.  A graph is denoted as $\g$, and $V(\g)$ and
$E(\g)$ are the node and edge sets of $\g$, respectively.  Since the
reachability information of nodes in a strongly connected component is identical, we assume directed acyclic graphs
(\DAG), for presentation brevity. A reachability query takes two nodes
$u$ and $v$ as input, denoted as $u \leadsto v$, and returns true {\it
  iff} $v$ is reachable from $u$.


\subsection{System Model}
\label{subsec:system_model}
We follow the system model that is
commonly used in the literature of database outsourcing, presented in
Fig.~\ref{fig:overview}.  The model consists of three parties:
%

\begin{itemize}%\addtolength{\itemsep}{-0.5\baselineskip}
% 1
\item {\em Data owner}: An owner owns the graph data and computes the
    privacy preserving \hop\ labeling offline {\em once}. It  outsources them to the
    service provider and delivers query clients a salt $s_2$ to encrypt queries
    and the secret key $K$ to decrypt results;  
% 2
\item {\em Service provider} (\SP): The \SP\ has high computational
  utility such as cloud computing. The \SP\ handles massive query
  requests over the encrypted data on behalf of the data owner and
  returns the encrypted results to clients; and
% 3
\item {\em Client}: A client encrypts a query using $s_2$, sends the
  encrypted query to the \SP\ and decrypts the result with
  $K$. We assume that the clients and \SP\ do not collude.
%
\end{itemize}


\subsection{Privacy Target}
Our privacy target is
required to keep the following two pieces of information private from {\em attackers}
using the two attacks defined in the attack model: 
%
\begin{itemize}%\addtolength{\itemsep}{-0.5\baselineskip}
    %
    \item {\em Reachability of the query nodes} a.k.a the query result. In
        particular, given a reachability query $u \leadsto v$, attackers cannot
        infer whether $u$ can reach to $v$; and
    %
    \item {\em Graph structure} a.k.a. the topology of the data graph, {\it
        e.g.}, the existence of an edge.
    %
\end{itemize} 


\subsection{Attack Model}
We assume the \SP s are {\em
  honest-but-curious}. The attackers  may be the
\SP\ or another adversary hacking the \SP.  For simplicity, we often {\em term the
  attackers as the \SP}. We assume that the \SP\ can adopt the
following attacks.\footnote{There are many other
  attacks in the literature.  We do not consider them all in
  one approach.}  
%
\begin{itemize}
    %
    \item {\em Ciphertext only attack}, where the \SP\ can
        access only the ciphertext (encrypted graph data) and does not know what
        their original graph is; and 
    %
    \item {\em Size based attack}, where the \SP\ attempts to infer the two pieces of private
        information from the sizes of the data and the query
        results.
    %
\end{itemize} 


To sum up, ...
