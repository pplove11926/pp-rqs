\section{Experimental Evaluation}
\label{sec:exp}

In this section, we present the experimental evaluation that verifies
the performance of our proposed techniques and the effectiveness of our
optimization.

\subsection{Experimental Setup}
\label{subsec:expsetup}


\noindent {\bf Running platform.} We conducted all experiments using a machine
with Intel Core i3-2310 2.10GHz CPU and 4G RAM running Windows 7 OS.  All
algorithms were implemented using ++ basCed on the implementation of \hop\
labelings provided by R.Bramandia et al. \cite{byron}. The hash function ($\hash_{s_1}$
and $\hash_{s_2}$) was  160-bit SHA-1 using two different salts. %and MD5 The
The encryption  $\encrypt$ was 1024-bit Elgamal \cite{elgamal}.

\stab
\noindent {\bf Datasets.} We used three synthetic datasets (denoted as \SYN) and four
real-world datasets. Some of their characteristics are shown in
Table~\ref{table:syndata}. The synthetic datasets were all scale-free graphs,
which are popular in experimentation. The generator used was provided by Choi
et al. \cite{generator}. We controlled the sizes and densities of the graphs by
setting $\alpha = 0.27$ and $\beta = 10$.  The real-world datasets are all
publicly available.\footnote{\scriptsize
YEAST: http://vlado.fmf.uni-lj.si/pub/networks/data/bio/Yeast/Yeast.htm \\
ODLTS: http://vlado.fmf.uni-lj.si/pub/networks/data/dic/odlis/Odlis.htm \\
ERDOS: http://vlado.fmf.uni-lj.si/pub/networks/data/Erdos/Erdos02.net \\
ROGET: http://vlado.fmf.uni-lj.si/pub/networks/data/dic/roget/Roget.htm
} % footnote of real data


