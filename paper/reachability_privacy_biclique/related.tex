\vspace{-2ex}
\section{Related Work}
\label{sec:related}
\vspace{-2ex}

In this section, we present some related works on security of
graph queries. 

%% We then introduce the current state of privacy preserving
%% graph query. We end this section with the introduction of the approaches for
%% reachability query.

\stab
\noindent{\bf Authentication of graph queries.} Query authentication
is a security problem where the \SP\ cannot be trusted. It requires the client to verify the correctness
of the data graphs returned.  Kundu et
al. \cite{auth_graph_wo_leaking,signature, forest} focus on the verification of the {\em authenticity} of a
{\em given portion of data} (subtree/subgraph that users' have the
right to access to) without leakage of extraneous information of the
data (tree/graph/forest). They optimize the signature needed
\cite{forest, signature}. 
Zhe et al. \cite{zhe} propose an efficient authenticated subgraph query
services framework under the outsourced graph databases system. 
In comparison, the input of our problem is a
reachability query not a subgraph for verification. Our problem
focuses on {\em protecting both the query and its answer from \SP s}.

\stab
\noindent{\bf Privacy preserving graph query.} He et
al. \cite{edgeprivacy} analyze the node reachability of the graph
data, with the preservation of edge privacy. Unfortunately, the method
reveals the reachability of the query nodes and partial structure of
the graph data to the \SP.  Gao et al.  \cite{Jeffery} propose
neighborhood-privacy protected shortest distance in the paradigm of
cloud computing. This aims to preserve all the neighborhood connections
and the shortest distances between each pair of nodes in outsourced
graph data. When this work is directly applied to reachability
queries, some information about the graph structure and the
reachability of the query nodes are still exposed to the \SP.

Mouratidis et al.  \cite{PIR+SP} determine the shortest path of the query
nodes with no information leakage by using the PIR \cite{PIR}
protocol.  Firstly, the high computational cost of PIR is still a
concern. Secondly, the PIR approach requires the transfer of the
same amount of data for every query, which can be large. Reachability
queries are simple (yet fundamental) queries and do not require the use of
the costly PIR method.  Cao et al. \cite{ICDCS} propose to
support subgraph query over an encrypted database of small graphs.
Their work protects the query privacy, index privacy and feature
privacy. However, reachability queries cannot be expressed as subgraph
queries. Karwa et al. \cite{subgraph+count} present an efficient
algorithm for outsourcing useful statistics of graph data by
protecting the edge {\em differential privacy} and they
support counting queries but not reachability queries.

\stab
\noindent{\bf Approaches for reachability
  query.}\label{subsec:related:reach} Numerous approaches for
reachability query have been proposed recently in the literature. Jin
et al. \cite{sigmod2012} recently discuss the approaches based on
three main categories: {\em transitive closure compressions}, {\em
  refined online search} and {\em hop labeling}. The seminal
\hop\ labeling is proposed by Cohen et al. \cite{cohen} and further optimized by many studies \cite{hopi,chengjf1, chengjf2}.  Due to the simple
structure of the \hop\ labeling approach and the simple query
evaluation, in this paper, we propose our techniques based on the
\hop\ labeling approach.


% 1

%% (i) Transitive closure compressions ({\it e.g.},\cite{sigmod1989,
%%   pathtree, bitcompression}) exhibit the best query
%% performances. It is known that their storage costs are the
%% highest and often impractical. 

%% (ii) Refined online search ({\it
%%   e.g.}, \cite{grail, gripp, ferrari}) utilizes some online
%% search strategies to answer reachability query. However, online
%% searches ({\it e.g.}, DFS and BFS) carry the information of
%% graph structures and it is not clear how  they can be adopted. 

%% (iii) The seminal \hop\ labeling is proposed by Cohen \cite{cohen}.
%% Various interesting works of \hop\ are recently proposed to enhance
%% query performances ({\it e.g.}, \cite{3hop, pathhop, sigmod2013})
%% and maintenance performances \cite{byron}. A stream of previous work
%% on \hop\ focuses on optimizing the sizes of \hop\ and its high
%% construction cost \cite{hopi,chengjf1, chengjf2}. Cheng et
%% al. \cite{sigmod2013} proposes a topological folding labeling scheme
%% that establishes an \hop\ like labeling.

%% Each vertex is assigned two sets of vertices (\hop\ labels)
%% only. Due to the simple structure of the \hop\ labeling approach and
%% the simple query evaluation, in this paper, we propose our techniques
%% based on the \hop\ labeling approach.




%% There are many studies on privacy preserving graph publication
%% undertaking the structure anonymization approach ({\it e.g.},
%% \cite{1-neighborhood, k-degree, k-automorphism,
%%   k-isomorphism}). Structural information of graphs are changed by
%% anonymization and it is not clear how to support reachability
%% queries. Due to space limitation, we do not include detailed
%% comparisons with these studies.

